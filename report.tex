% xelatex
% vim:set et sw=4 ts=4 tw=80:
\documentclass[12pt]{report}

\usepackage{setspace}
\usepackage{fontspec}
\usepackage[a4paper, margin=1in]{geometry}
\usepackage{blindtext}
\usepackage{indentfirst}
\usepackage[
    backend=biber,
    style=apa,
    defernumbers=true,
]{biblatex}
\addbibresource{report.bib}
\usepackage{hyperref}
\usepackage{lipsum}
\usepackage{graphicx}

\setstretch{1.15}
% \setmainfont{Times}
\title{Comparing the Qualitative Effects Between Procrastination and Family
Pressure on Exam Anxiety}
\author{
    SalahDin Ahmed Salh Rezk
    \href{mailto:salahdin.1519013@stemluxor.moe.edu.eg}{{\fontspec{Symbola}\char"1F4E7}}
    \and
    Younis Tarek Hanafi Metwaly
    \href{mailto:younis.1519035@stemluxor.moe.edu.eg}{{\fontspec{Symbola}\char"1F4E7}}
    \and
    Marawan MogebElrahman Foud AbdElbaset
    \href{mailto:marawan.1519030@stemluxor.moe.edu.eg}{{\fontspec{Symbola}\char"1F4E7}}
    \\\\
    Luxor STEM School
    \includegraphics[height=.85em]{images/luxor.jpg}
    and
    Qena STEM School
    \includegraphics[height=.85em]{images/qena.png}
    \\\\
    English Class Grade 12
    \\\\
    Mr. Khalil Na3ama
    % \href{mailto:khalil@stemqena.moe.edu.eg}{{\fontspec{Symbola}\char"1F4E7}}
}
\date{\today}

\begin{document}
\maketitle
\tableofcontents

\abstract{
    some abstract, hehe
}

\chapter{Problem}
\section{Introduction}

It is a notable phenomenon that individuals do experience some kind of anxiety
throughout their life — especially their academic one — increasing the
probability of mistakes during most critical tasks. Although there is no
possible advantage to such trait, it is prominently visible in the majority of
whom are considered mentally stable by modern physiological standards
\parencite{Leake1946-cw}. It is usually the case that such seemingly useless
traits affecting people's day-to-day life has some kind of an advantage to
humans' ancestors \parencite{Price2003-bl}. Therefore, the pursuit of analyzing
this condition is a critical pivot for the advancement of human's development
sciences.

Anxiety can be described as the tense, unsettling anticipation of a threatening
but vague event; a feeling of uneasy suspense \parencite{Rachman2019-cn}. There
is, noticeably, two distinct types of anxiety: objective and neurotic. Fear is
usually considered as a type of objective anxiety, while neurotic anxiety is a
product of internal perceptions and emotions \parencite{Spielberger1966-dk}.
Although objective anxiety is more considered the helpful one for the humans'
survival, neurotic anxiety is more often considered the ineffective one for
the modern age — the interest of ours.

According to the \cite{mowrer1939stimulus}, the neurotic anxiety is defined as a
result of the act which an individual commits to but wishes they had not. This
definition, although may be misleading, does explain the frequency of such a
feeling; it encourages the individual to be more aware of the consequences of
their commitments. This can be noticed through most modern subcategories of
anxiety (e.g. exam anxiety, marriage anxiety, etc). In contrast to  \cite[which
suggests anxiety to be an extension of the parenting
instinct]{sullivan2013interpersonal}, that cannot explain modern forms of
anxiety yet can be explained as a form of objective anxiety. Therefore,
Mowerer's model could be more helpful for the purposes of this study.

Attention should be directed towards the field of education due to its flexible
nature compared to other professional fields, also it is the foundation for
every single considerable profession thus it is the focus in this study.
Education, however, is such a broad subject that cannot be contained within one
piece of work; accordingly, the most concerning of factors shall be analysed. It
was decided that exams, and hence exams anxiety, are the most serious due to the
both their effect on one's mental state and possible future.

\section{Impact}

To assess the true value of the proposed problem the quantitative impact should
be investigated to avoid wasting resources on an unnecessary matter. The main
focus will be the effect on academic performance, although the observation of
long term results would have been better, but there is no sufficient resources
for such a task. The exclusion of possible variable is an important
consideration since the special circumstances of COVID-19 have unnaturally
affected the results of previous years due to the unusual increase in stress
resulting in a less-stable mental state for most students
\parencite{covid19-impact}.

\cite{rana_2010_the} describes the effect of cognitive factors (i.e worry) on
academic performance of the students. The impact of anxiety was indisputable;
the worry scale had the greatest correlation with the students' performance
compared to all other measured scales. In more details, \cite{trifoni2011does}
dives more into the specific different states relating to anxiety, a one
illustrates the students' opinions on the topic: students who were more anxious
had more radical opinions concerning the subject compared to their less anxious
peers.

\section{Possible Factors}

There is, clearly, a number of measurable factors that results in exam anxiety.
An incomprehensive list of them include: load of courses, duration of exams,
expectations of exams, control over exams, procrastination, physical exercise,
quality of rests, confidence, nutritions, and time management Some will be
discussed here in order to form a coherent image for the current state of this
problem. Despite that, the goal of this section is to decide the factors that
should be discussed in order to reach a clear conclusion for this study.

Even though course load was conducted to be the most important factor, it is the
only exam-related affective factor \parencite{hashmat2008factors}. All the other
sufficiently significant factors are directly related to the psychological
state of the student (e.g. self-confidence, expectations, etc), ergo the focus on
psychological health aspects. 

\chapter{Hypothesis}

\lipsum

\chapter{Methodology}

\lipsum

\chapter{Findings}

\lipsum

\chapter{Conclusion}

\lipsum

\printbibliography[heading=bibnumbered]

\chapter{Appendix}

\section{Questionnaire}
\lipsum
\section{Previous Corrected Versions}
\lipsum

\end{document}
