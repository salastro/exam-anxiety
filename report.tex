% xelatex
% vim:set et sw=4 ts=4 tw=80:
% \documentclass[12pt]{apa7}
\documentclass[12pt]{report}

\usepackage{setspace}
\usepackage{fontspec}
\usepackage[a4paper, margin=1in]{geometry}
\usepackage{blindtext}
\usepackage{indentfirst}
\usepackage[
    backend=biber,
    style=apa,
    defernumbers=true,
]{biblatex}
\addbibresource{report.bib}
\usepackage{hyperref}
\usepackage{lipsum}
\usepackage{graphicx}
% \usepackage{hologo}
% \usepackage{pgfplots}
\usepackage{multicol}

% \pgfplotsset{compat=1.17}
% \usepgfplotslibrary{external}
% \tikzexternalize

\setstretch{1.15}
% \setmainfont{Doulos SIL}[
%     ItalicFont = {EB Garamond-Italic},
% ]
% \setmainfont{EB Garamond}
% \setmainfont{Source Serif Pro}
% \setmainfont{PT Serif}
% \setmainfont{Charis SIL}
\setmainfont{Times New Roman}
\pagenumbering{roman}
\title{Comparing the Qualitative Effects Between Procrastination and Family
Pressure on Exam Anxiety}
\author{
    SalahDin Ahmed Salh Rezk
    \href{mailto:salahdin.1519013@stemluxor.moe.edu.eg}{{\fontspec{Symbola}\char"1F4E7}}
    \and
    Younis Tarek Hanafi Metwaly
    \href{mailto:younis.1519035@stemluxor.moe.edu.eg}{{\fontspec{Symbola}\char"1F4E7}}
    \and
    Marawan MogebElrahman Foud AbdElbaset
    \href{mailto:marawan.1519030@stemluxor.moe.edu.eg}{{\fontspec{Symbola}\char"1F4E7}}
    \\\\
    Luxor STEM School
    \includegraphics[height=.85em]{images/luxor.jpg}
    and
    Qena STEM School
    \includegraphics[height=.85em]{images/qena.png}
    \\
    under the Egyptian Ministry of Education
    \includegraphics[height=.85em]{images/ministry.png}
    \\\\
    English Class Grade 12
    \\\\
    Mr. Khalil Na3ama
    % \href{mailto:khalil@stemqena.moe.edu.eg}{{\fontspec{Symbola}\char"1F4E7}}
}
\date{\today}

\renewcommand{\abstractname}{Executive Summary}

\begin{document}
\maketitle
\tableofcontents
\listoffigures
\listoftables

\abstract{
    Exam anxiety is a huge factor in the daily life of students. It may
    affect their possible performance in exams resulting in an undesirable future.
    The most common causes of exam anxiety are the lack of preparation and family
    pressure. This report presents the qualitative effects of two in order to form
    more sophisticated plans for dealing with possible educational issues. The
    method by which these factors are compared is based on a 5-dimensional 100-point
    scale whose main input is a questionnaire done by a sample of 43 students. The
    model concluded that the procrastination has more effect than family pressure on
    exam anxiety by 43\% with a 0.048 p-value. This model, however, does not
    consider the performance of the students as an important factor hence the need
    of more development for its structure.
    \addcontentsline{toc}{chapter}{Executive Summary}
}
\thispagestyle{plain}

\setcounter{page}{1}
\pagenumbering{arabic}

\chapter{Introduction}
\begin{multicols}{2}
\section{Problem}

It is a notable phenomenon that individuals do experience some kind of anxiety
throughout their life — especially their academic one — increasing the
probability of mistakes during most critical tasks. It is usually the case that
such seemingly useless traits affecting people's day-to-day life has some kind
of an advantage to humans' ancestors \parencite{Price2003-bl}. 

Anxiety can be described as the tense, unsettling anticipation of a threatening
but vague event \parencite{Rachman2019-cn}. There is two distinct types of
anxiety: objective and neurotic. Fear is usually considered as a type of
objective anxiety, while neurotic anxiety is a product of internal perceptions
and emotions \parencite{Spielberger1966-dk}. 

According to the \cite{mowrer1939stimulus}, neurotic anxiety is defined as a
result of the act which an individual commits to but wishes they had not. This
definition does explain the frequency of such a feeling; this can be noticed
through most modern subcategories of anxiety (e.g. exam anxiety, marriage
anxiety, etc).

Attention should be directed towards the field of education due to its flexible
nature compared to other professional fields, also it is the foundation for
every single considerable profession. The main type of anxiety concerning this
field is exam anxiety \parencite{academic-anxiety}, ergo the focus of this study.

\cite{rana_2010_the} describes the effect of cognitive factors (i.e worry) on
academic performance of the students. The impact of anxiety was indisputable;
the worry scale had the greatest correlation with the students' performance
compared to all other measured scales. In addition, \cite{trifoni2011does} works
on different scales relating to anxiety, a one illustrates the students'
opinions on the topic: students who were more anxious had more radical opinions
concerning the subject compared to their less anxious peers.

\section{Hypothesis}

The main focus of the study is the effects
of procrastination and family pressure on exam anxiety. Accordingly, the
proposed hypothesis is that procrastination does have a higher effect than that
of family pressure. The importance of such hypothesis may not seem clear at
first; however, the result will determine which should be the concern of a
family dealing with their children. \textit{``Should I change the way I am
supporting my kid or force him to study more and avoid procrastination?"} is the
kind of questions that this research try to answer effectively.

\end{multicols}

\chapter{Body}
\begin{multicols}{2}
    
\section{Methodology}

\subsection{Sample}

The sample were mainly consisted of randomized students ranging from high school
to university. The students were from different geographic areas, different
schools, and different systems of education. The sample was taken from the
Egyptian internet population, so the results will be limited to the Egyptian
population at best. The age of the students ranged from 15 to 22; although there
were some outliers ranging from 30 to 49. The gender was not collected due to
its overall insignificant affect on the results \parencite{hashmat2008factors}.

\subsection{Questions}

\section{Findings}

\lipsum

\end{multicols}


\chapter{Conclusion}
\begin{multicols}{2}

\section{Summary of Findings}
\lipsum
\section{Recommendations}

\lipsum

\end{multicols}

\printbibliography[heading=bibnumbered]

\chapter{Appendix}
\section{Questionnaire}
\lipsum
% \appendix
% \chapter{Questionnaire}
% \lipsum
% \chapter{Technical Notes}

% This document was completely made within {\LaTeX} markup language for
% typesetting with \hologo{BibTeX} for bibliography using an external text editor
% i.e. NeoVIM while managing version revisions and teamwork with Git.
\end{document}
