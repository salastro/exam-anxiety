% xelatex
% vim:set et sw=4 ts=4 tw=80:
% \documentclass[12pt]{apa7}
\documentclass[12pt]{report}

\usepackage{setspace}
\usepackage{fontspec}
\usepackage[a4paper, margin=1in]{geometry}
\usepackage{blindtext}
\usepackage{indentfirst}
\usepackage[
    backend=biber,
    style=apa,
    defernumbers=true,
]{biblatex}
\addbibresource{report.bib}
\usepackage{hyperref}
\usepackage{lipsum}
\usepackage{graphicx}
\usepackage{hologo}

\setstretch{1.15}
% \setmainfont{Doulos SIL}[
%     ItalicFont = {EB Garamond-Italic},
% ]
% \setmainfont{EB Garamond}
% \setmainfont{Source Serif Pro}
% \setmainfont{PT Serif}
% \setmainfont{Charis SIL}
\setmainfont{Times New Roman}
\pagenumbering{roman}
\title{Comparing the Qualitative Effects Between Procrastination and Family
Pressure on Exam Anxiety}
\author{
    SalahDin Ahmed Salh Rezk
    \href{mailto:salahdin.1519013@stemluxor.moe.edu.eg}{{\fontspec{Symbola}\char"1F4E7}}
    \and
    Younis Tarek Hanafi Metwaly
    \href{mailto:younis.1519035@stemluxor.moe.edu.eg}{{\fontspec{Symbola}\char"1F4E7}}
    \and
    Marawan MogebElrahman Foud AbdElbaset
    \href{mailto:marawan.1519030@stemluxor.moe.edu.eg}{{\fontspec{Symbola}\char"1F4E7}}
    \\\\
    Luxor STEM School
    \includegraphics[height=.85em]{images/luxor.jpg}
    and
    Qena STEM School
    \includegraphics[height=.85em]{images/qena.png}
    \\
    under the Egyptian Ministry of Education
    \includegraphics[height=.85em]{images/ministry.png}
    \\\\
    English Class Grade 12
    \\\\
    Mr. Khalil Na3ama
    % \href{mailto:khalil@stemqena.moe.edu.eg}{{\fontspec{Symbola}\char"1F4E7}}
}
\date{\today}

\renewcommand{\abstractname}{Executive Summary}

\begin{document}
\maketitle
\tableofcontents
\listoffigures
\listoftables

\abstract{
    Exam anxiety is a huge factor in the daily life of students. It may
    affect their possible performance in exams resulting in an undesirable future.
    The most common causes of exam anxiety are the lack of preparation and family
    pressure. This report presents the qualitative effects of two in order to form
    more sophisticated plans for dealing with possible educational issues. The
    method by which these factors are compared is based on a 5-dimensional 100-point
    scale whose main input is a questionnaire done by a sample of 43 students. The
    model concluded that the procrastination has more effect than family pressure on
    exam anxiety by 43\% with a 0.048 p-value. This model, however, does not
    consider the performance of the students as an important factor hence the need
    of more development for its structure.
}
\thispagestyle{plain}

\setcounter{page}{1}
\pagenumbering{arabic}

\chapter{Introduction}
\section{Problem}
\subsection{Literature}

It is a notable phenomenon that individuals do experience some kind of anxiety
throughout their life — especially their academic one — increasing the
probability of mistakes during most critical tasks. Although there is no
possible advantage to such trait, it is prominently visible in the majority of
whom are considered mentally stable by modern physiological standards
\parencite{Leake1946-cw}. It is usually the case that such seemingly useless
traits affecting people's day-to-day life has some kind of an advantage to
humans' ancestors \parencite{Price2003-bl}. Therefore, the pursuit of analyzing
this condition is a critical pivot for the advancement of human's development
sciences.

Anxiety can be described as the tense, unsettling anticipation of a threatening
but vague event; a feeling of uneasy suspense \parencite{Rachman2019-cn}. There
is, noticeably, two distinct types of anxiety: objective and neurotic. Fear is
usually considered as a type of objective anxiety, while neurotic anxiety is a
product of internal perceptions and emotions \parencite{Spielberger1966-dk}.
Although objective anxiety is more considered the helpful one for the humans'
survival, neurotic anxiety is more often considered the ineffective one for
the modern age — the interest of ours.

According to the \cite{mowrer1939stimulus}, the neurotic anxiety is defined as a
result of the act which an individual commits to but wishes they had not. This
definition, although may be misleading, does explain the frequency of such a
feeling; it encourages the individual to be more aware of the consequences of
their commitments. This can be noticed through most modern subcategories of
anxiety (e.g. exam anxiety, marriage anxiety, etc). In contrast to  \cite[which
suggests anxiety to be an extension of the parenting
instinct]{sullivan2013interpersonal}, that cannot explain modern forms of
anxiety yet can be explained as a form of objective anxiety. Therefore,
Mowerer's model could be more helpful for the purposes of this study.

Attention should be directed towards the field of education due to its flexible
nature compared to other professional fields, also it is the foundation for
every single considerable profession thus it is the focus in this study.
Education, however, is such a broad subject that cannot be contained within one
piece of work; accordingly, the most concerning of factors shall be analysed. It
was decided that exams, and hence exams anxiety, are the most serious due to the
both their effect on one's mental state and possible future.

\subsection{Impact}

To assess the true value of the proposed problem the quantitative impact should
be investigated to avoid wasting resources on an unnecessary matter. The main
focus will be the effect on academic performance, although the observation of
long term results would have been better, but there is no sufficient resources
for such a task. The exclusion of possible variable is an important
consideration since the special circumstances of COVID-19 have unnaturally
affected the results of previous years due to the unusual increase in stress
resulting in a less-stable mental state for most students
\parencite{covid19-impact}.

\cite{rana_2010_the} describes the effect of cognitive factors (i.e worry) on
academic performance of the students. The impact of anxiety was indisputable;
the worry scale had the greatest correlation with the students' performance
compared to all other measured scales. In more details, \cite{trifoni2011does}
dives more into the specific different states relating to anxiety, a one
illustrates the students' opinions on the topic: students who were more anxious
had more radical opinions concerning the subject compared to their less anxious
peers.

\subsection{Possible Factors}

There is, clearly, a number of measurable factors that results in exam anxiety.
An incomprehensive list of them include: load of courses, duration of exams,
expectations of exams, control over exams, procrastination, physical exercise,
quality of rests, confidence, nutritions, and time management Some will be
discussed here in order to form a coherent image for the current state of this
problem. Despite that, the goal of this subsection is to decide the factors that
should be discussed in order to reach a clear conclusion for this study.

Even though course load was conducted to be the most important factor, it is the
only exam-related affective factor \parencite{hashmat2008factors}. All the other
sufficiently significant\footnote{In statistical hypothesis testing, a result
has statistical significance when it is very unlikely to have occurred.} factors
are directly related to the psychological state of the student (e.g.
self-confidence, expectations, etc), ergo the focus on psychological health
aspects. However, an aspect that is greatly is usually overlooked is pressure
caused by whom around the individual (e.g. family pressure, peer pressure, etc).
It can affect a key factor, self-esteem \parencite{whitbeck1991family}.

While procrastination\footnote{Procrastination is also referred to as
preparation in some studies.} is a known yet not-well studied factor, pressure
is not considered in most studies as a factor of any significance. Because of
that the focus will be to compare a known and less known factors in order to
result in a more comprehensive view of the issue in hand. However, the subject
of pressure is an unmanageably broad concept that needs to be reduced into a
feasible subcategory. \cite{whitbeck1991family} considers family pressure by far
the most significant emotional factor affecting self-esteem and confidence hence
the focus will be on it.

\section{Hypothesis}

\subsection{Defining}

As mentioned in the previous section, the main focus of the study is the effects
of procrastination and family pressure on exam anxiety. Accordingly, the
proposed hypothesis is that procrastination does have a significantly higher
effect than that of family pressure. The importance of such hypothesis may not
seem clear at first and might even rise more issues; however, the result will
determine which should be the concern of a family dealing with their children.
\textit{``Should I change the way I am supporting my kid or force him to study
more and avoid procrastination?"} is the kind of questions that this research
try to answer effectively.

It is important to note that the hypothesis is not a direct comparison between
the two points of failure i.e. self and family, but rather a concise overview on
the metrics that are being investigated. Furthermore, the result of the study is
a subset of the problem. The main benefit will be the possible plans that could
be derived. Thus, the formal statement that of the null hypothesis\footnote{In
inferential statistics, the null hypothesis is that two possibilities are the
same.} is that there is no statistically significant difference between the
negative effects of procrastination and family pressure on exam anxiety, while
the alternative hypothesis\footnote{In statistical hypothesis testing, the
alternative hypothesis is one of the proposed propositions.} is that there is a
statistically significant difference between the negative effects of
procrastination.

\subsection{Testing}

The approach to these kind of psychological metrics is usually based on detailed
questionnaires with studied designs to achieve the goal of qualitatively
measuring the focused on metrics \parencite{jack1998purpose}. Although the use
of interest-based surveys is the most common due to its ease and its concise
simple random samples\footnote{In statistics, a simple random sample is a subset
of a larger set where they are chosen randomly.}, it presents some notable
limitations \parencite{wright2005researching}. The main limitation that may
affect the accuracy of this study is the lack of honesty and consistency of the
online population. \cite{lying} found that participant it is way easier to lie
when you put the internet's mask of anonymity. Looking decent is more important
for them than the accuracy of scientific research.

In order to minimize this type of issues, the questionnaire should be both
passive (to avoid question bias) and unintimidating (to avoid social
desirability bias). Each questions should be phrased in a way that is not
intrusive to the reader. It should decrease the level of judgement towards any
subset of people and illustrates that there is no problem in being radically
different. Another method that may help is the process of making the
questionnaire more realistic than usual online surveys; this may help in making
the readers be more aware of the importance of the results hence being more
honest. Despite all of that, demanded characteristics bias may still be present
in the questionnaire because of their nature relating to human perception of the
examination itself.

\chapter{Body}
\section{Methodology}

\subsection{Sample}

\subsection{Questions}

\section{Findings}

\lipsum

\chapter{Conclusion}
\section{Summary of Findings}
\section{Recommendations}

\lipsum

\printbibliography[heading=bibnumbered]


% \chapter{Appendix}
% \section{Questionnaire}
\appendix
\chapter{Questionnaire}
\lipsum
\chapter{Previous Corrected Versions}
\lipsum
\chapter{Technical Notes}

This document was completely made within {\LaTeX} markup language for
typesetting with \hologo{BibTeX} for bibliography/references using an external
text editor i.e. NeoVim while managing version revisions with Git.

\end{document}
