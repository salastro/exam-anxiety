% xelatex
% vim:set et sw=4 ts=4 tw=80:
% \documentclass[12pt]{apa7}
\documentclass[12pt]{report}

\usepackage{setspace}
\usepackage{fontspec}
\usepackage[a4paper, margin=1in]{geometry}
\usepackage{blindtext}
\usepackage{indentfirst}
\usepackage[
    backend=biber,
    style=apa,
    defernumbers=true,
]{biblatex}
\addbibresource{report.bib}
\usepackage{hyperref}
\usepackage{kantlipsum}
\usepackage{lipsum}
\usepackage{graphicx}
% \usepackage{hologo}
\usepackage{pgfplots}
\usepackage{tikz,tikz-3dplot}
\usepackage{float}
\usepackage{multicol}
\usepackage{enumitem}
\usepackage{amsmath}
% \usepackage{datatool}
% \DTLloaddb{data}{data/data-nums.csv}
\usepackage{pgf-pie}  

\pgfplotsset{compat=newest}
% \usepgfplotslibrary{external}
% \tikzexternalize


\setstretch{1.15}
% \setmainfont{Doulos SIL}[
%     ItalicFont = {EB Garamond-Italic},
% ]
% \setmainfont{EB Garamond}
% \setmainfont{Source Serif Pro}
% \setmainfont{PT Serif}
% \setmainfont{Charis SIL}
\setmainfont{Times New Roman}
\pagenumbering{roman}
\title{Comparing the Qualitative Effects Between Procrastination and Family
Pressure on Exam Anxiety}
\author{
    Marawan MogebElrahman Foud AbdElbaset \\
    \texttt{\href{mailto:marawan.1519030@stemluxor.moe.edu.eg}{marawan.1519030@stemluxor.moe.edu.eg}} \\
    SalahDin Ahmed Salh Rezk \\
    \texttt{\href{mailto:salahdin.1519013@stemluxor.moe.edu.eg}{salahdin.1519013@stemluxor.moe.edu.eg}} \\
    Younis Tarek Hanafi Metwaly \\
    \texttt{\href{mailto:younis.1519035@stemluxor.moe.edu.eg}{younis.1519035@stemluxor.moe.edu.eg}} \\
    \\\\
    Luxor STEM School
    \includegraphics[height=.85em]{images/luxor.jpg}
    and
    Qena STEM School
    \includegraphics[height=.85em]{images/qena.png}
    \\
    under the Egyptian Ministry of Education
    \includegraphics[height=.85em]{images/ministry.png}
    \\\\
    English Class Grade 12
    \\\\
    Mr. Khalil Na3ama
    % \href{mailto:khalil@stemqena.moe.edu.eg}{{\fontspec{Symbola}\char"1F4E7}}
}
\date{\today}

\renewcommand{\abstractname}{Executive Summary}

\begin{document}
\maketitle
\tableofcontents
\listoffigures
% \listoftables

\abstract{
    Exam anxiety is a huge factor in the daily life of students. It may affect
    their possible performance in exams resulting in an undesirable future. The
    most common causes of exam anxiety are the lack of preparation and family
    pressure. This report presents the qualitative effects of two in order to
    form more sophisticated plans for dealing with possible educational issues.
    The method by which these factors are compared is based on a 3-dimensional
    10-point scale whose main input is a questionnaire done by a sample of 43
    students. The model concluded that the procrastination has more effect than
    family pressure on exam anxiety by 43\% with a 0.048 p-value. This model,
    however, does not consider the performance of the students as an important
    factor hence the need of more development for its structure.
    \addcontentsline{toc}{chapter}{Executive Summary}
}
\thispagestyle{plain}

\setcounter{page}{1}
\pagenumbering{arabic}

\chapter{Introduction}
\begin{multicols}{2}
\section{Problem}

It is a notable phenomenon that individuals do experience some kind of anxiety
throughout their life — especially their academic one — increasing the
probability of mistakes during most critical tasks. It is usually the case that
such seemingly useless traits affecting people's day-to-day life has some kind
of an advantage to humans' ancestors \parencite{Price2003-bl}. 

Anxiety can be described as the tense, unsettling anticipation of a threatening
but vague event \parencite{Rachman2019-cn}. There is two distinct types of
anxiety: objective and neurotic. Fear is usually considered as a type of
objective anxiety, while neurotic anxiety is a product of internal perceptions
and emotions \parencite{Spielberger1966-dk}. 

According to the \cite{mowrer1939stimulus}, neurotic anxiety is defined as a
result of the act which an individual commits to but wishes they had not. This
definition does explain the frequency of such a feeling; this can be noticed
through most modern subcategories of anxiety (e.g. exam anxiety, marriage
anxiety, etc).

Attention should be directed towards the field of education due to its flexible
nature compared to other professional fields, also it is the foundation for
every single considerable profession. The main type of anxiety concerning this
field is exam anxiety \parencite{academic-anxiety}, ergo the focus of this study.

\cite{rana_2010_the} describes the effect of cognitive factors (i.e worry) on
academic performance of the students. The impact of anxiety was indisputable;
the worry scale had the greatest correlation with the students' performance
compared to all other measured scales. In addition, \cite{trifoni2011does} works
on different scales relating to anxiety, a one illustrates the students'
opinions on the topic: students who were more anxious had more radical opinions
concerning the subject compared to their less anxious peers.

\section{Hypothesis}

The main focus of the study is the effects
of procrastination and family pressure on exam anxiety. Accordingly, the
proposed hypothesis is that procrastination does have a higher effect than that
of family pressure. The importance of such hypothesis may not seem clear at
first; however, the result will determine which should be the concern of a
family dealing with their children. \textit{``Should I change the way I am
supporting my kid or force him to study more and avoid procrastination?"} is the
kind of questions that this research try to answer effectively.

\end{multicols}

\chapter{Body}
\begin{multicols}{2}
    
\section{Methodology}

\subsection{Sample}

The sample were mainly consisted of randomized students ranging from high school
to university. The students were from different geographic areas, different
schools, and different systems of education. The sample was taken from the
Egyptian internet population, so the results will be limited to the Egyptian
population at best. The age of the students ranged from 15 to 22; although there
were some outliers ranging from 30 to 49. The gender was not collected due to
its overall insignificant affect on the results \parencite{hashmat2008factors}.

\subsection{Questions}

\begin{enumerate}[wide, labelwidth=!, labelindent=0pt]

    \item \textit{How much did procrastination affect your academic performance?}
\textbf{(10-point scale)}

This question measures the affect of procrastination on performance from the
student's perspective essentially measuring the amount of procrastination
while avoiding the feel of guilt this question usually results.

\item \textit{How harsh did your family go on you to study last year?}
\textbf{(10-point scale)}

This question measures the amount of family pressure a student has endured without
implicitly mentioning family pressure to avoid bias caused by family relations
with mentioning it more as a beneficial parental act.

\item \textit{How satisfied were you with your last year's academic
performance?}
\textbf{(10-point scale)}

This question measures the student's satisfaction with their academic
performance in order to correlate it with procrastination and family pressure.

\item \textit{How stressful did you feel about last year's exams?}
\textbf{(10-point scale)}

This question measures the student's stress factor, which affects exam anxiety
the most, in order to correlate it with procrastination and family pressure.

\item \textit{How hard it takes you to recover after bad grades?}
\textbf{(open-ended)}

This question tries to find the relation between the amount of exam anxiety
and the recovering factor.

\item \textit{What are the factors that lead to exam anxiety from your
perspective?}
\textbf{(open-ended)}

This questions collects other factors that may affect or be affected by family
pressure, procrastination, or both. This may even help in drawing conclusions
beyond the scope of this study.

\item \textit{How do you get ready mentally for your exams?}
\textbf{(open-ended)}

This question tries to find solutions provided by students and correlated them
with the amount of exam anxiety in order to measure their effectiveness.

\item \textit{Which affected your performance the most?}
\textbf{(multiple choices)}

This question tests student's perspective of their situation to enable comparing
what students think with the calculated results of other questions.

\end{enumerate}

\section{Findings}

There is a notable correlation between both family pressure and procrastination
seen in Figure~\ref{fig:procrastination-stress} and
Figure~\ref{fig:family-stress}. Despite that, there is a weak correlation in
Figure~\ref{fig:procrastination-family} that may hint to problems within this
model. Thus, a more sophisticated tool should be formulated; accordingly, the
exam anxiety factor ($EA$) is defined in Equation~\ref{eq:exam-anxiety}.

A 3-dimensional plot of the exam anxiety could be helpful in demonstrating
the relation between procrastination and family pressure. Figure~\ref{fig:3d}
shows a greater correlation between the two factors and exam anxiety, but family
pressure is clearly more dominant. The increase of both factors results in a
noticeable increase in exam anxiety.

The best fit curve for family pressure ($F$) and procrastination ($P$) are
Equation~\ref{eq:eaf-f} and Equation~\ref{eq:eaf-p} respectively with
determination coefficients of $R^2 = 0.04$ and $R^2 = 0.06$.

\begin{equation}
    EAF = \textnormal{stress} -
    \frac{\textnormal{satisfaction}}{\textnormal{recovery}}
    \label{eq:exam-anxiety}
\end{equation}

\begin{figure}[H]
    \begin{tikzpicture}
        \begin{axis}[
            width = \linewidth,
            only marks,
            mark size=3pt,
            scatter,
            xlabel = procrastination,
            ylabel = stress,
            legend style={
                at={(0.44,-0.2)},
                anchor=north,legend columns=-1
            },
            grid style=dashed,
            ymajorgrids=true,
            xmajorgrids=true,
            ]
            \addplot[] table [
                x=procrastination,
                y=stress,
                col sep=comma
                ]{data/data-nums.csv};
        \end{axis}
    \end{tikzpicture}
    \caption{Procrastination versus stress plot}
    \label{fig:procrastination-stress}
\end{figure}

\begin{figure}[H]
    \begin{tikzpicture}
        \begin{axis}[
            width = \linewidth,
            only marks,
            mark size=3pt,
            scatter,
            xlabel = family pressure,
            ylabel = stress,
            legend style={
                at={(0.44,-0.2)},
                anchor=north,legend columns=-1
            },
            grid style=dashed,
            ymajorgrids=true,
            xmajorgrids=true,
            ]
            \addplot[] table [
                x=family,
                y=stress,
                col sep=comma
                ]{data/data-nums.csv};
        \end{axis}
    \end{tikzpicture}
    \caption{Family pressure versus stress plot}
    \label{fig:family-stress}
\end{figure}

\begin{figure}[H]
    \begin{tikzpicture}
        \begin{axis}[
            width = \linewidth,
            only marks,
            mark size=3pt,
            scatter,
            xlabel = procrastination,
            ylabel = family pressure,
            legend style={
                at={(0.44,-0.2)},
                anchor=north,legend columns=-1
            },
            grid style=dashed,
            ymajorgrids=true,
            xmajorgrids=true,
            ]
            \addplot[] table [
                x=procrastination,
                y=family,
                col sep=comma
                ]{data/data-nums.csv};
        \end{axis}
    \end{tikzpicture}
    \caption{Procrastination versus family pressure plot}
    \label{fig:procrastination-family}
\end{figure}

\begin{align}
    \label{eq:eaf-f}
    EA &= 0.027 F^3 - 0.4 F^2 + 1.6 F + 3\\
       \label{eq:eaf-p}
       &= -0.20 P^2 + 2.7 P - 2.7\\
       \label{eq:eaf-fp}
       &= 0.18 F - 0.20 P + 5.6
\end{align}

Despite that, the unmatching Equation~\ref{eq:eaf-fp} has a determination
coefficient $R^2 = 0.024$, which may indicate that the two factors have a strong
mutual impact on exam anxiety. Additionally, these equations indicate that,
while procrastination has a larger affect, it does not stay as long as family
pressure, and after a specific critical point ($P=7$) its effect starts
declining.

On the other hand, Figure~\ref{fig:bar} shows the total of exam anxiety across
age groups. The bar chart shows a clear increase around the age of 17, which is
the average age for high school seniors. This reflects the large importance that
the Egyptian society puts on this year resulting in an in increase in exam
anxiety. At the age of 20 students tend to be more anxious than at the age of
15, although there is less pressure on their academic life as they are in
college. This may indicate that other factors take affect on exam anxiety later
on in life.

\begin{figure}[H]
    \begin{tikzpicture}
        \begin{axis}[
            width = \linewidth,
            mark size=2.7pt,
            zlabel = exam anxiety,
            grid style=dashed,
            ymajorgrids=true,
            xmajorgrids=true,
            zmajorgrids=true,
            zlabel style={sloped},
            clip=false,
            ]
            \addplot3[
                scatter,
                only marks,
                ] table [
                x=procrastination,
                y=family,
                z=EAF,
                col sep=comma,
                ]{data/data-nums.csv};
            \node at (rel axis cs:0,0.4,1.1) [above,sloped like y axis] {family
                pressure};
            \node at (rel axis cs:1,0,1.7) [above,sloped like x axis]
                {procrastination};
            % \addplot3 [
            %     mesh,
            %     samples=4,
            %     domain=0:10,
            %     ]{0.178141*y-0.208792*x+5.63211};
        \end{axis}
    \end{tikzpicture}
    \caption{Exam anxiety plot}
    \label{fig:3d}
\end{figure}

\begin{figure}[H]
    \begin{center}
        \begin{tikzpicture}[]
            \begin{axis}[
                ybar stacked,
                ]
                \addplot+[
                    ybar,
                    ] table[
                    x=age,
                    y=sum EAF,
                    col sep=comma,
                    ] {data/bar.csv};
            \end{axis}
        \end{tikzpicture}
    \end{center}
    \caption{Bar chart of exam anxiety by age}
    \label{fig:bar}
\end{figure}

\end{multicols}

\chapter{Conclusion}
\begin{multicols}{2}

\section{Summary of Findings}

\section{Recommendations}

\end{multicols}

\printbibliography[heading=bibnumbered]

\chapter{Appendix}
\section{Questionnaire}
% \appendix
% \chapter{Questionnaire}
% \lipsum
% \chapter{Technical Notes}

% This document was completely made within {\LaTeX} markup language for
% typesetting with \hologo{BibTeX} for bibliography using an external text editor
% i.e. NeoVIM while managing version revisions and teamwork with Git.
\end{document}
